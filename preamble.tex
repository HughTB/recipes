% Setup basic page geometry
\usepackage{geometry}

\geometry{
  a4paper,
  total={180mm, 257mm},
  left=15mm,
  top=20mm
}

% Command to insert cover page
% Document-Title
\newcommand{\coverpage}[1]{
  \begin{titlepage}
    \pagecolor{myPurple}
    \color{white}

    \vspace*{\fill}

    \noindent \Huge{\textit{#1}}\\[2mm]
    \noindent \huge{Hugh Baldwin}\\[1mm]
    \Large{contact@hughtb.uk}

    \vspace*{\fill}
  \end{titlepage}
  \nopagecolor{}
}

% Setup hyperlink colours
\usepackage[]{hyperref}
\hypersetup{
  colorlinks=true,
  linkcolor=myPurple,
  urlcolor=myPurple
}

% Basic packages
\usepackage[no-math]{fontspec}
\usepackage{fontawesome}

% Set headers and footers, as well as PDF metadata
\usepackage{lastpage}
\usepackage{fancyhdr}
\usepackage{hyperref}

\renewcommand{\chaptermark}[1]{ \markboth{#1}{} }
\renewcommand{\sectionmark}[1]{ \markright{#1}{} }

% Document-Title Github-Repo
\newcommand{\setupdocument}[2]{
  % Override the styling on a plain page
  \fancypagestyle{plain}{
    \renewcommand{\headrulewidth}{0pt}

    \fancyfoot[L]{Hugh Baldwin \\ \small{\url{#2}}}
    \fancyfoot[C]{\textbf{\thepage} of \pageref*{LastPage}}
    \fancyfoot[R]{#1}

    \fancyhead[L]{\leftmark}
    \fancyhead[R]{}
  }

  \pagestyle{fancy}

  \renewcommand{\headrulewidth}{0pt}

  \fancyfoot[L]{Hugh Baldwin \\ \small{\url{#2}}}
  \fancyfoot[C]{\textbf{\thepage} of \pageref*{LastPage}}
  \fancyfoot[R]{#1}

  \fancyhead[L]{\leftmark}
  \fancyhead[R]{}

  % Fancyhdr complains without changing the header height
  \setlength{\headheight}{14.5pt}
  \addtolength{\topmargin}{-2.5pt}

  \hypersetup{
    pdftitle={#1},
    pdfauthor={Hugh Baldwin}
  }
}

% Set the format of chapter headers using titlesec
\usepackage{titlesec}

\titleformat{\chapter}[display]
  {\normalfont\bfseries\Huge\centering}{}{0pt}{}

\newcommand{\difficulty}[1]{
  \foreach \x in {1,...,5}{\ifnumgreater{\x}{#1}{\faStarO}{\faStar}}
}

% Command to create a new recipe (chapter)
% Name Time Diet Difficulty Servings Ingredients Method
\newcommand{\recipe}[7]{
  \chapter{#1}

  \vspace{-35pt}
  
  \hrule

  \vspace{10pt}

  \begin{center}
    \begin{tabular}{ p{0.2\linewidth} p{0.2\linewidth} p{0.2\linewidth} p{0.2\linewidth} }
      \centering{\faClockO\ Time\\#2} &
      \centering{\faLeaf\ Diet\\#3} &
      \centering{\faStarHalfEmpty\ Difficulty\\\difficulty{#4}} &
      \centering{\faChild\ Servings\\#5} \\
    \end{tabular}
  \end{center}

  \begin{center}
    \begin{tabular}{ p{0.25\linewidth} p{0.7\linewidth} }
      {
        \textbf{Ingredients}
    
        \begin{raggedright}
          \begin{itemize}[label={},leftmargin=*]
            #6
          \end{itemize}
        \end{raggedright}
      } & {
        \textbf{Method}
    
        \begin{raggedright}
          \begin{enumerate}
            #7
          \end{enumerate}
        \end{raggedright}
      } \\
    \end{tabular}
  \end{center}
}

% Use the xcolor package, and add my custom purple
\usepackage[dvipsnames]{xcolor}

\definecolor{myPurple}{RGB}{134, 53, 227}

% Use the amsmath package so normal text can be used within math blocks
\usepackage{amsmath}

% Use the mathspec package to set main and math fonts
\usepackage{mathspec}

\setmainfont[Ligatures=TeX]{IBM Plex Sans}
\setsansfont[Ligatures=TeX]{IBM Plex Sans}
\setmonofont[Ligatures=TeX]{IBM Plex Mono}
\setmathfont(Digits,Latin){IBM Plex Sans}

% Use the tikz package for diagrams
\usepackage{tikz}

\tikzset{
  vertex/.style={circle,thick,draw,inner sep=0pt,minimum size=6.5mm}
}

% Custom enumerations
\usepackage{enumitem}

% Diagonal fractions
\usepackage{xfrac}